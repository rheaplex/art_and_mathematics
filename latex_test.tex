\documentclass[11pt]{article}
\usepackage{geometry}                % See geometry.pdf to learn the layout options. There are lots.
\usepackage{exscale,relsize} 
\usepackage{color}
\geometry{letterpaper}                   % ... or a4paper or a5paper or ... 
%\geometry{landscape}                % Activate for for rotated page geometry
%\usepackage[parfill]{parskip}    % Activate to begin paragraphs with an empty line rather than an indent
\usepackage{graphicx}
\usepackage{amssymb}
\usepackage{epstopdf}
\DeclareGraphicsRule{.tif}{png}{.png}{`convert #1 `dirname #1`/`basename #1 .tif`.png}

\newcommand\Red{{\color{red} \blacksquare}}
\newcommand\Yellow{{\color{yellow} \blacksquare}}
\newcommand\Blue{{\color{blue} \blacksquare}}
\newcommand\Primaries{{\mathbb C}}

\title{Brief Article}
\author{The Author}
%\date{}                                           % Activate to display a given date or no date

\begin{document}
\maketitle
%\section{}
%\subsection{}

\section {Colour}

$\Primaries = \left\{ \Blue, \Red, \Yellow \right\}$

${\mathbb C}' = \left\{ {\color{orange} \blacksquare }, {\color{purple} \blacksquare }, {\color{green} \blacksquare } \right\}$



\section {Mathematical Equations}

Learning latex maths markup.

$
 \textcolor{red}{\bullet{}} 
$

$
{\color{blue} \blacksquare }
$


$
\cos{x}=0\Leftrightarrow x\in\{(n+$$\frac{1}{2}$$)\pi:n\in\mathbb{Z}\}
$



$
\left[\frac{\left(\frac{x+3}{7}\right)+5}{3x+8}\right] 
$

$x_1 = y^2 = z^4_1$

$x_{ijk}$

$\left( \frac{n+2}{3} \right)$

$\left\{ \frac{ n^2 + n}{ n^2 - n} \right\}$

$\in , \infty ,\exists, \forall, \Rightarrow ,\leftrightarrow , \pi ,\sum,\int$


$
f(x)=\int_{-\infty}^x\frac{(e^{-t^2})}{\sqrt{\pi^x}}dt
$




$
\mathcal{P}(X)
$

$
M^2 = \left(
\begin{array}{cc}
M^2_{11} & M^2_{18}\\
M^2_{18} & M^2_{88}
\end{array}
\right)
$


$
f(x) = \sqrt{1+x} \quad (x \ge -1)
f(x) = \sqrt{1+x}, \quad x \ge -1
f(x) \sim x^2 \quad (x\to\infty)
f(x) \sim x^2, \quad x\to\infty
$

Simple equations, like $x^y$ or $x_n = \sqrt{a + b}$ can be typeset right
in the text line by enclosing them in a pair of single dollar sign symbols.
Don't forget that if you want a real dollar sign in your text, like \$2000,
you have to use the \verb+\$+ command.

A more complicated equation should be typeset in {\em displayed math\/} mode,
like this:
\[
z \left( 1 \ +\ \sqrt{\omega_{i+1} + \zeta -\frac{x+1}{\Theta +1} y + 1}
\ \right)
\ \ \ =\ \ \ 1
\]
The ``equation'' environment displays your equations, and automatically
numbers them consecutively within your document, like this:
\begin{equation}
\left[
{\bf X} + {\rm a} \ \geq\
\underline{\hat a} \sum_i^N \lim_{x \rightarrow k} \delta C
\right]
\end{equation}


\[ \sum \neq \mathlarger{\sum} \] 
and $\frac{1}{2} \neq \frac{\mathlarger 1} 
{2}$ but $N = \mathlarger {N}$ 


\[ \mathsmaller\sum_{i=1}^n \neq 
\sum_{i=1}^n \neq \mathlarger\sum_{i=1}^n 
\qquad \mathsmaller\int_0^\infty \neq 
\int_0^\infty \neq \mathlarger\int_0^\infty 
\] 


\[ a \times b \neq a \mathlarger{\times} b \neq 
a \mathbin{\mathlarger\times} b \]


\end{document}



\end{document}  