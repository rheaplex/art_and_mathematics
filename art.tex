%%%%%%%%%%%%%%%%%%%%%%%%%%%%%%%%%%%%%%%%%%%%%%%%%%
% Setup
%%%%%%%%%%%%%%%%%%%%%%%%%%%%%%%%%%%%%%%%%%%%%%%%%%
\documentclass[11pt]{article}
\usepackage{geometry}                % See geometry.pdf to learn the layout options. There are lots.
\usepackage{exscale,relsize} 
\usepackage{color}
\geometry{a4paper}                   % ... or a4paper or a5paper or ... 
%\geometry{landscape}                % Activate for for rotated page geometry
\usepackage[parfill]{parskip}    % Activate to begin paragraphs with an empty line rather than an indent
\usepackage{graphicx}
\usepackage{amssymb}
\usepackage{epstopdf}
\DeclareGraphicsRule{.tif}{png}{.png}{`convert #1 `dirname #1`/`basename #1 .tif`.png}

%%%%%%%%%%%%%%%%%%%%%%%%%%%%%%%%%%%%%%%%%%%%%%%%%%
% New graphics entities
%%%%%%%%%%%%%%%%%%%%%%%%%%%%%%%%%%%%%%%%%%%%%%%%%%

\newcommand\power{\mathcal{P}}

\newcommand\Colour{\mathsmaller{\blacksquare}}

\newcommand\Red{{\color{red} \Colour}}
\newcommand\Yellow{{\color{yellow} \Colour}}
\newcommand\Blue{{\color{blue} \Colour}}

\newcommand\Orange{{\color[rgb]{1.0, 0.5, 0.0}  \Colour}}
\newcommand\Green{{\color{green} \Colour}}
\newcommand\Purple{{\color[rgb]{0.5, 0.0, 0.5}  \Colour}}

\newcommand\YellowOrange{{\color[rgb]{1.0, 0.75, 0.0} \Colour}}
\newcommand\RedOrange{{\color[rgb]{1.0, 0.25, 0.0} \Colour}}
\newcommand\RedPurple{{\color[rgb]{1.0, 0.0, 0.5} \Colour}}
\newcommand\BluePurple{{\color[rgb]{0.75, 0.0, 1.0}\Colour}}
\newcommand\YellowGreen{{\color[rgb]{0.75, 1.0, 0.0} \Colour}}
\newcommand\BlueGreen{{\color[rgb]{0.0, 1.0, 0.75} \Colour}}

\newcommand\Black{{\color{black} \Colour}}
\newcommand\White{{\color{black} \square}}
\newcommand\Grey{{\color[rgb]{0.5, 0.5, 0.5} \Colour}}

\newcommand\Primaries{C}
\newcommand\Secondaries{C^{\prime}}
\newcommand\Tertaries{C^{\prime{}\prime{}}}
\newcommand\Achromatics{\neg{}C}

\newcommand\PointShape{{\cdot}}
\newcommand\LineShape{{\mid}}
\newcommand\TriangleShape{{\bigtriangleup}}
\newcommand\CircleShape{{\bigcirc}}
\newcommand\SquareShape{{\Box}}

\newcommand\Shapes{S}

\newcommand\VerticalLine{{\mid}}
\newcommand\HorizontalLine{\textit{---}}
\newcommand\DiagonalUpLine{{/}}
\newcommand\DiagonalDownLine{{\setminus}}
\newcommand\Lines{L}

%%%%%%%%%%%%%%%%%%%%%%%%%%%%%%%%%%%%%%%%%%%%%%%%%%
% Document body
%%%%%%%%%%%%%%%%%%%%%%%%%%%%%%%%%%%%%%%%%%%%%%%%%%

\title{Art}
\author{Rob Myers}
%\date{}                                           % Activate to display a given date or no date

\begin{document}
\maketitle

%%%%%%%%%%%%%%%%%%%%%%%%%%%%%%%%%%%%%%%%%%%%%%%%%%
% Notation
%%%%%%%%%%%%%%%%%%%%%%%%%%%%%%%%%%%%%%%%%%%%%%%%%%

\section{Notation}

$<$ - Numeric less than.

$>$ - Numeric more than.

$\wedge$ - Logical and.

$\vee$ - Logical or.

$\not$ - Logical not

$\in$ - Set membership.

$\cup$ - Set union. 

$\cap$ - Set Intersection.

$\setminus$ - Set difference.

$\#{}$ - Set cardinality.

$=$ - Numeric or set equality.

$\sum $- Numeric or aesthetic sum.

$(a, b)$ - An ordered pair. Also representable as a set using XXXX's notation: \\
    $\left\{ \left\{ a \right\}, \left\{ a, b \right\} \right\}$ \\
We use this notation when manipulating the elements of an ordered pair.

$\left\{ a, b \right\}$ - A set. The set consists of the listed elements.

$\left\{ A \mid B \bullet C \right\}$ - A set comprehension. 
The members of the set are the result of the expression C, 
which is applied to every value of the variables introduced in A for which B is true.

The notation used is that of "The Essence Of Computing: Discrete mathematics", XXXX, YEAR, PUBLISHER.

\pagebreak

%%%%%%%%%%%%%%%%%%%%%%%%%%%%%%%%%%%%%%%%%%%%%%%%%%
% Ordered Pairs
%%%%%%%%%%%%%%%%%%%%%%%%%%%%%%%%%%%%%%%%%%%%%%%%%%

\section{Ordered Pairs}

\subsection{The First And Second Items Of An Ordered Pair}

An ordered pair can be represented as a set.

$p = (1, 2) = \left\{\left\{1\right\}, \left\{1,2\right\}\right\}$

The first set of the pair is the same as the intersection of the first and second member sets.

$\left\{1\right\} \cap \left\{1, 2\right\} = \left\{1\right\}$

$\cap \left\{\left\{1\right\}, \left\{1,2\right\}\right\} = \left\{1\right\}$

$\cap p = \cap \left\{\left\{1\right\}, \left\{1,2\right\}\right\} = \left\{1\right\}$

The second set of the pair is the same as the union of the first and second member sets.

$\left\{1\right\} \cup \left\{1, 2\right\} = \left\{1, 2\right\}$

$\cup \left\{\left\{1\right\}, \left\{1,2\right\}\right\} = \left\{1, 2\right\}$

$\cup p = \cap \left\{\left\{1\right\}, \left\{1,2\right\}\right\} = \left\{, 21\right\}$

The first item of the pair is the only item of the intersection of the first and second member sets.

$X = (\cap p) \cap (\cup p) = \left\{1\right\} \cap \left\{1, 2\right\} = 1$

$x \in X$

$x = 1$

The second item of the pair is the difference of the second member set with the first member set.

$Y = (\cap p)  \setminus (\cup p)  = \left\{1, 2\right\} \setminus \left\{1\right\} = \left\{1\right\}$

$y \in Y$

$y = 2$

The sum of the pair is the sum of the second member set of the pair.

$z = \sum \cup p$

$Z = \cup \left\{\left\{1\right\}, \left\{1,2\right\}\right\} = \left\{1, 2\right\}$

$z = \sum Z = 1 + 2 = 3$

\pagebreak

\subsection{Sets of Ordered Pairs}

We can sum each ordered pair in a set.

$\left\{x :\in X \bullet \sum x\right\}$

We can get all the first items of a set of pairs

$S = \left\{(1,2), (3, 4), (5, 6)\right\}$

$T = \left\{ x :\in S \bullet \cup x \right\} = \left\{1, 3, 5\right\}$

And all the second items

$V = \left\{(1,2), (3, 4), (5, 6)\right\}$

$W = \left\{ x :\in S \bullet \\ x \right\} = \left\{2, 4, 6\right\}$

Some items in some pairs may be the same.

$P = \left\{(1,2), (1, 4), (5, 4)\right\}$

$Q = \left\{ x : \in S \bullet \cup x \right\} = \left\{1, 1, 5\right\}$

So the cardinality may be different.

$\#{}S = \#{}T$

$\neg{}(\#{}P = \#{}Q)$

If the cardinality of the set of all the first items is equal to the cardinality of the set of pairs, each first item is unique.

$\#{} U = \#{}\ S$

If the cardinality of the set of all the first items is less than the cardinality of the set of pairs, not every first item is unique.

$\neg{}(\#{} R = \#{} P)$

We can get every member of a power set of ordered pairs with unique first items.

$A = \left\{\left\{(1,2), (1, 4), (5, 4)\right\}, \left\{(1,2), (1, 4), (5, 4)\right\}\right\}$

$B = \left\{a :\in A | \#{}a = (\#{}\left\{ x in a \bullet \cup x \right\})\right\}$

And we can sum that

$A = \left\{\left\{(1,2), (1, 4), (5, 4)\right\}, \left\{(1,2), (1, 4), (5, 4)\right\}\right\}$

$B = \left\{a \in A | \#{}a = (\#{}\left\{ x \in a \bullet \cup x \right\}) \#{}\right\}$

%Ax \in X E1 y \in X ^ \\ x = \\ y
%For every x in X there is precisely one y in X whose intersection equals, it i.e. that it is the only one with that intersection and hence that first item.

\pagebreak

%%%%%%%%%%%%%%%%%%%%%%%%%%%%%%%%%%%%%%%%%%%%%%%%%%
% Colour
%%%%%%%%%%%%%%%%%%%%%%%%%%%%%%%%%%%%%%%%%%%%%%%%%%

\section {Colour}
\subsection{Primaries}
$\Primaries = \left\{ \Blue, \Red, \Yellow \right\}$ \\
$\Blue < \Red < \Yellow$

\subsection{Secondaries}
$\Secondaries = \left\{ \Purple, \Green, \Orange \right\}$ \\
$\Purple < \Green < \Orange$

\subsection{Tertiaries}
$\Tertaries = \left\{ \YellowOrange, \RedOrange, \RedPurple, \BluePurple, \YellowGreen, \BlueGreen \right\}$ \\
$\YellowOrange < \RedOrange < \RedPurple < \BluePurple < \YellowGreen < \BlueGreen$

\subsection{Achromatics}
$\Achromatics = \left\{ \Black, \Grey, \White \right\}$ \\
$\Black < \Grey < \White$

%%%%%%%%%%%%%%%%%%%%%%%%%%%%%%%%%%%%%%%%%%%%%%%%%%
% Shape
%%%%%%%%%%%%%%%%%%%%%%%%%%%%%%%%%%%%%%%%%%%%%%%%%%

\section{Shape}

$\Shapes = \left\{  \PointShape, \CircleShape, \LineShape, \TriangleShape, \SquareShape \right\}$

$\Lines = \left\{  \VerticalLine, \DiagonalUpLine, \HorizontalLine, \DiagonalDownLine  \right\}$

\section{Shape and Colour}

$\SquareShape + \Blue = {\color{blue} \blacksquare}$ \\
${\color{blue} \blacksquare} - \Blue = \SquareShape$ \\
$\CircleShape + \Red = {\color{red} \bullet}$ \\
${\color{red} \bullet} - \Red = \CircleShape$

\pagebreak

%%%%%%%%%%%%%%%%%%%%%%%%%%%%%%%%%%%%%%%%%%%%%%%%%%
% Sol LeWitt
%%%%%%%%%%%%%%%%%%%%%%%%%%%%%%%%%%%%%%%%%%%%%%%%%%

\section{Sol LeWitt}

\subsection{Straight lines in four directions \& all their possible combinations}
$\Lines = \left\{  \VerticalLine, \DiagonalUpLine, \HorizontalLine, \DiagonalDownLine \right\}$ \\
$M = \power{}\Lines$ \\
$N = \left\{ x \in M \bullet \sum x \right\}$

\subsection{All one-, two-, three- \& four part combinations of four transparent colours}
$\Primaries = \left\{ \Blue, \Red, \Yellow, \Black \right\}$ \\
$D = \power \Primaries$ \\
$E = \left\{ x \in D \bullet \sum x \right\}$ % \mid \#{}x < 5 

\subsection{All one-, two-, three- \& four part combinations of lines in four directions and in four colours}
$C = \left\{ \Red, \Yellow, \Blue, \Black \right\} $ \\
$L = \left\{ \mid, \\, /, - \right\} $ \\
$V = C \times L = \left\{ (\Red, \mid), \ldots (\Black, -) \right\} $ \\
$W = PV = \left\{ \left\{ (\Red, \mid) \right\} , \ldots \left\{ (\Red, \mid), \ldots (\Black, -) \right\}  \right\} $ \\
$X = \left\{ x : \in W \mid \#{}x > 0 \and \#{}x < 5 \right\}$ \\  % All sets of 1..4
$Y = \left\{ x : \in X \mid \#{}x = \#{} \cup (\cup x)  \right\} $ \\ % All sets with unique first pair items
$Z = \left\{ x : \in Y, y : \in  x \bullet \cup (\cup y) \right\} $ \\ % Each item of each member set combined, note that | applies � to *every* value

Alternative Z:

$Z = \left\{ x : \in Y bullet \cup (\cup x) \right\}$ \\  % Each member set combined

% So what can we express that LeWitt didn't do?

\pagebreak

\subsection{Various lengths of representation for ``All one-, two-, three- \& four part combinations of lines in four directions and in four colours''}

$3: x \in X$

$4: x: \in X$

$5: \left\{ x \in X \right\} , x: \in \power C$

$6: \left\{ x \in \power C \right\} , x \in X \bullet \sum x$

$7: x : \in X \bullet \sum x , \left\{ x : \in \power C \right\} , 
      x \in \power C \bullet \sum x , \power \left\{ k y r b \right\}$

$8: \left\{ x \in X \bullet \sum x \right\}$

$9: \left\{ x : \in X \bullet \sum x \right\}$

$10: Y = \left\{ x \in X \bullet \sum x \right\}$

$11: Y = \left\{ x : \in X \bullet \sum x \right\}, Y = \left\{ x \in \power C \bullet \sum x \right\}$

$12: Y = \left\{ x : \in \power C \bullet \sum x \right\}$

$15: x \in \power \left\{ k, r, y, b \right\} \bullet \sum x , \left\{ x : \in \power \left\{ k r y b \right\} \bullet \sum x \right\}$

\pagebreak

%%%%%%%%%%%%%%%%%%%%%%%%%%%%%%%%%%%%%%%%%%%%%%%%%%
% Damian Hirst
%%%%%%%%%%%%%%%%%%%%%%%%%%%%%%%%%%%%%%%%%%%%%%%%%%

\section{Damien Hirst}

\subsection{Spot Paintings}

$s(x) = \exists y | x = \frac{1}{2}(2y^{2})$ \\ % Square number
$A = \power \left\{ hsv \right\}$ \\
$B = \left\{ x \in A  \mid s(\#{} x)\right\}$ \\
$D = \left\{ x \in B \bullet x + \CircleShape \right\}$

\pagebreak

%%%%%%%%%%%%%%%%%%%%%%%%%%%%%%%%%%%%%%%%%%%%%%%%%%
% Godel Numbering
%%%%%%%%%%%%%%%%%%%%%%%%%%%%%%%%%%%%%%%%%%%%%%%%%%

\section{Godel Numbering}

Godel Numbering, named after Austrian mathematician Kurt Godel, is the encoding of mathematical statements into numbers by some arbitrary scheme. The scheme used here is derived from that used by Douglas Hofstadter in his "Godel, Escher, Bach - An Eternal Golden Braid", YEAR, PUBLISHER.

Several post-conceptual artists have used numbers as the content of their artworks. Godel Numbering a set representation of an artwork refers to this among other things.

\subsection{Notation}

Using a variant of Hofstadter's scheme.

\{ = 11 \\
\} = 12 \\
$, = 13$ \\
$\mid  = 21$ \\
$\bullet = 22$ \\
$= = 23$ \\
$\in = 31$ \\
$\power = 41$ \\
$\#{} = 42$ \\
$< = 43$ \\
$\sum = 44$ \\
$x = 51$ \\
$\Primaries = 61$ \\
$D = 74$ \\
$E = 75$ \\
$\Blue = 81$ \\
$\Red = 82$ \\
$\Yellow = 82$ \\
$\Black = 84$ \\
$hsv = 89$ \\
$0 =  91$ \\
$S = 92$ \\

\subsection{All one-, two-, three- \& four part combinations of four transparent colours}

$612311811382138313841312$

$74236141641$

$7523115131742142514392929292929122445112$

or

$1151311181138213831384131241118113821383138413122142514392929292929122445112$

\pagebreak

%%%%%%%%%%%%%%%%%%%%%%%%%%%%%%%%%%%%%%%%%%%%%%%%%%
% Saville Colouring
%%%%%%%%%%%%%%%%%%%%%%%%%%%%%%%%%%%%%%%%%%%%%%%%%%

\section{Saville Colouring}

Saville colouring, named here after British graphic designer Peter Saville, is the encoding of numeric or textual information into arbitrary sequences of colours. Such a scheme was used by Saville on a number of record covers for Factory Records in the 1980s, with the key being placed on the back cover of New Order's "Power, Corruption and Lies", XXXX.

Many artists have used arbitrary grids or other arrays of arbitrary colours as the content of their artworks. Saville Colouring a set representation of an artwork refers to this among other things.

\subsection{Colours}


\end{document}
